
%-------------------------------------------------------------------------
%	QUOTATION PAGE
%-------------------------------------------------------------------------
\quotepage{Alan Turing}
{
	The original question, ‘Can machines think?’ I believe to be too meaningless to deserve discussion.
}

%-------------------------------------------------------------------------
%	ACKNOWLEDGEMENTS PAGE
%-------------------------------------------------------------------------
\begin{acknowledgements}
\begin{otherlanguage}{portuguese}
Esta tese foi financiada pela Fundação para a Ciência e a Tecnologia (FCT), no âmbito da bolsa de doutoramento SFRH/BD/129600/2017.

Gostaria de expressar o meu agradecimento ao INESC TEC, instituição de acolhimento do meu doutoramento, à Faculdade de Ciências da Universidade do Porto, pela oferta do programa doutoral, e à Faculdade de Engenharia da Universidade do Porto, por me ter proporcionado a honra e o enorme prazer de lecionar duas unidades curriculares de um dos seus planos de estudos.

A nível pessoal, manifesto a minha profunda gratidão ao meu orientador, Prof.\ Jaime S.\ Cardoso, inexcedível no apoio, disponibilidade, supervisão, paciência e compreensão em momentos difíceis, qualidades a que alia excelência científica inquestionável e amplamente reconhecida. Não posso também esquecer os meus professores Paula Malonek e Pedro Guedes de Oliveira, que muito me ajudaram e inspiraram no meu percurso académico anterior e me apoiaram e aconselharam na fase complicada que antecedeu o meu doutoramento.

Todos os atuais e anteriores membros do VCMI merecem a minha maior consideração e, em muitos casos, inestimável amizade. Trabalhar ao vosso lado e conviver convosco fez de mim uma pessoa melhor, tanto no plano profissional como pessoal. Um agradecimento especial aos coautores das minhas publicações, cujo contributo para esta tese foi absolutamente fundamental. Entre estes, tenho de destacar o Pedro Ferreira, não só pelo extenso trabalho que desenvolvemos em conjunto, mas também e sobretudo por ser um amigo insubstituível em todos os momentos.

Agradeço a todos os meus estudantes na FEUP, em particular àqueles que hoje tenho como amigos. Todas as horas que passámos, em aula e fora dela, ensinaram-me muito mais do que aquilo que vos lecionei. Espero que continuem a contar comigo e poder continuar a contar convosco.

Aos meus amigos de sempre e que nunca me falham, estando perto ou longe.

À Rita, por estar sempre comigo e me fazer desejar que assim seja para sempre. À minha família, o meu grande suporte. À minha mãe, pelo seu amor infinito e incondicional.

\vspace{11pt}

Ao meu pai e avós, com saudade imensa.

\end{otherlanguage}
\end{acknowledgements}

\addvspacetoc{0.3cm} % Add a gap in the Contents, for aesthetics


%-------------------------------------------------------------------------
%	ABSTRACT PAGE (PORTUGUESE)
%-------------------------------------------------------------------------
\begin{abstract}[
	thesistitle={Aprendizagem a partir de dados multi-entidade},
	title={Resumo},
	degree={Doutoramento em Ciência de Computadores},
	nameconnector={por}]
\begin{otherlanguage}{portuguese}

A maioria dos algoritmos de aprendizagem automática, em particular os concebidos para classificação supervisionada, assume que os dados nos quais irão ser treinados resultam de um processo de amostragem independente e identicamente distribuída. Este pressuposto ignora o facto de que, em muitos casos, o conjunto de treino disponível consiste numa coleção de subconjuntos, cada um deles proveniente da sua própria distribuição. O mesmo se aplica aos dados utilizados para teste e inferência, cuja distribuição associada difere frequentemente da observada durante o treino do modelo.

Nesta tese, abordamos, sob várias perspetivas, o problema de aprendizagem a partir de múltiplas distribuições. Em primeiro lugar, consideramos a situação em que várias fontes de dados coabitam no mesmo meio e originam sequências de dados. Neste contexto, demonstramos que é possível tirar partido das correlações existentes entre as várias fontes, obtendo-se um modelo generativo para estes dados cujo desempenho é superior ao resultante da aprendizagem a partir de cada uma das fontes separadamente.

Posteriormente, dedicamo-nos ao problema da generalização fora da distribuição de treino, isto é, ao desenvolvimento de modelos robustos quando testados em distribuições substancialmente diferentes da observada durante o treino. Inicialmente, consideramos o problema da adaptação de domínio a partir de múltiplas distribuições-fonte. Neste cenário, assume-se que, na fase de treino, estão disponíveis dados anotados provenientes de múltiplas distribuições-fonte e que se pretende obter um modelo de classificação com bom desempenho numa distribuição-alvo, fixa à partida, mas diferente de todas as distribuições-fonte e para a qual apenas estão disponíveis dados não anotados. No âmbito deste tema, investigamos diversas arquiteturas de redes neuronais para contagem de objetos em vídeos provenientes de múltiplas câmaras e propomos um novo algoritmo de adaptação de domínio para dados não sequenciais cujo desempenho supera o do estado da arte.

Finalmente, focamo-nos na generalização de domínio. Este problema difere da adapta-\allowbreak ção de domínio no sentido em que, desta vez, o domínio-alvo não é definido \emph{a priori}, pelo que, na fase de treino, não estão disponíveis quaisquer dados provenientes da distribuição-\allowbreak-alvo. O reconhecimento de língua gestual é a principal aplicação que utilizamos para motivar o problema e os algoritmos desenvolvidos, visto que um sistema de reconhecimento gestual realmente útil deve apresentar um bom desempenho independentemente do gestuante que o utiliza. Para este efeito, propomos dois algoritmos que permitem a obtenção de um classificador de gestos robusto a novos gestuantes. O primeiro destes consiste numa melhoria de um método pré-existente, baseado em redes neuronais adversárias. O segundo, que é também o que apresenta o melhor desempenho, utiliza um codificador automático variacional, que permite a aprendizagem de representações latentes independentes do gestuante e altamente discriminativas para a tarefa de classificação pretendida.

Esta tese é, assim, um trabalho abrangente na área de aprendizagem multi-entidade e multi-domínio. Como explicado anteriormente, tratamos vários problemas que se adequam a diferentes aplicações do mundo real, que frequentemente utilizamos para motivar e validar os algoritmos propostos. As contribuições aqui apresentadas resultaram em diversas publicações em revistas e conferências internacionais, que listamos no decorrer deste documento.


\end{otherlanguage}
\end{abstract}

%-------------------------------------------------------------------------
%	ABSTRACT PAGE
%-------------------------------------------------------------------------
\begin{abstract}

    Most machine learning algorithms, and particularly those conceived for supervised classification, assume that the data on which they are trained are independent and identically distributed (i.i.d.). This assumption ignores the fact that, in many circumstances, the available training data is a collection of sub-datasets, each one being originated at a distinct data source and hence being sampled from its own distribution. Furthermore, the data observed at test time often suffer from the same distribution shift problem, again violating the i.i.d.\ assumption.

    In this thesis, we address the problem of learning from multiple data distributions under various settings. First, we consider the case where multiple data sources are spread over a given medium and produce sequential data streams. Although the data distributions for each of the sources are different, we show that inter-source correlations can be exploited to learn a better-performing model for the generative distribution of each source.

    Afterward, we focus on out-of-distribution generalization, i.e.\ to the setting in which the test distribution is unknown or only partially known at training time. Initially, we consider the problem of multi-source domain adaptation. Here, annotated data from multiple data sources are combined to learn a classification model for a fixed target distribution, for which no labeled data is available at training time. We research several possible deep neural network architectures for object counting in videos from multiple cameras and propose a novel algorithm for multi-source domain adaptation for non-sequential data that achieves state-of-the-art results.

    Finally, we address domain generalization. This problem differs from domain adaptation in that the target domain is unknown, and therefore no data from this distribution is available at training time, neither labeled nor unlabeled. Sign language recognition is the main application considered here since a truly useful automatic sign language recognition system should be able to perform accurately regardless of the signer that is using it. We propose two algorithms to learn a signer-independent sign classifier. The first is an improvement over an existing method based on adversarial neural networks and the second and best-performing uses a variational autoencoder to learn highly discriminative signer-invariant representations.

    This thesis is then a broad-scope work on the topic of multi-entity and multi-domain learning. As explained before, we address multiple problem formulations that fit different practical applications, which are often used to motivate and validate the proposed algorithms. The contributions presented in this thesis resulted in publications in international journals and conferences, listed later in this document.

\end{abstract}

%-------------------------------------------------------------------------
%	LIST OF CONTENTS/FIGURES/TABLES
%-------------------------------------------------------------------------

\addvspacetoc{0.3cm}
\setcounter{tocdepth}{3}
\tableofcontents % Write out the Table of Contents

\listoffigures % Write out the List of Figures

%\listoftables % Write out the List of Tables

%\addvspacetoc{0.3cm}

%-------------------------------------------------------------------------
%	PHYSICAL CONSTANTS/OTHER DEFINITIONS
%-------------------------------------------------------------------------

%\begin{listofcontants}
%	\const{My little ponny test of magical rainbow}{$mn/mp$}
%    {$2.997\ 924\ 58\times10^{8}\ \mbox{ms}^{-\mbox{s}}$}
%   \const{Vaccuum permeability test of magical rainbow for a specific case of
%   condensed matter physics}
%   {$\epsilon_0$}{$2.997\ 924\ 58\times10^{8}\ \mbox{ms}^{-\mbox{s}}$}
%	\const{Speed of Light test of magical rainbow}{$c$}
%    {$2.997\ 924\ 58\times10^{8}\ \mbox{ms}^{-\mbox{s}}$}
%\end{listofcontants}


%-------------------------------------------------------------------------
%	SYMBOLS
%-------------------------------------------------------------------------

%\begin{listofsymbols}
%	\symb{$F_{\mu\nu}$}{Maxwell tensor}{F}
%	\symb{$a$}{distance}{m}
%	\\
%	\symb{$\omega$}{angular frequency}{rads$^{-1}$}
%\end{listofsymbols}


%-------------------------------------------------------------------------
%	NOTATION
%-------------------------------------------------------------------------

\newcommand\notationname{Notation and Conventions}
\addtotoc{\notationname}
\fancyhead[LO]{\textsc{\notationname}}

\input{Notation}



%-------------------------------------------------------------------------
%	ABBREVIATIONS
%-------------------------------------------------------------------------

\begin{glossary}
    \abbrev{AP}{access point}
    \abbrev{CNN}{convolutional neural network}
    \abbrev{CPD}{conditional probability distribution}
    \abbrev{CVAE}{conditional variational autoencoder}
    \abbrev{DA}{domain adaptation}
    \abbrev{DeSIRe}{deep signer invariant representations (model name)}
    \abbrev{DG}{domain generalization}
    
    \abbrev{ELBO}{evidence lower bound}
    \abbrev{EM}{expectation-maximization}
    \abbrev{HMM}{hidden Markov model}
    \abbrev{LSTM}{long short-term memory}
    \abbrev{KL}{Kullback-Leibler}
    \abbrev{MAE}{mean absolute error}
    \abbrev{MAP}{maximum a posteriori}
    \abbrev{MDAN}{multi-source domain adversarial
networks}
    \abbrev{ME}{movement of epenthesis}
    \abbrev{MHMM}{mixture of hidden Markov models}
    \abbrev{MKLM}{Microsoft Kinect and Leap Motion (dataset)}
    \abbrev{MLLR}{maximum likelihood linear regression}
    \abbrev{MLP}{multi-layer perceptron}
    \abbrev{MODA}{multi-source mildly optimistic domain adaptation (model name)}
    \abbrev{MODA-FM}{MODA with FixMatch regularization (model name)}
    \abbrev{OOD}{out-of-distribution}
    \abbrev{PAD}{presentation attack detection}
    \abbrev{PAI}{presentation attack instrument}
    \abbrev{PAIS}{presentation attack instrument species}
    \abbrev{ReLU}{rectifier linear unit}
    \abbrev{RNN}{recurrent neural network}
    \abbrev{SOHMMM}{self-organizing hidden Markov model map}
    \abbrev{SOM}{self-organizing map}
    \abbrev{SpaMHMM}{sparse mixture of hidden Markov models}
    \abbrev{SLR}{sign language recognition}
    \abbrev{SSL}{self-supervised learning}
    \abbrev{STA}{wireless station}
    \abbrev{SVM}{support vector machine}
    \abbrev{t-SNE}{t-distributed stochastic neighbor embeddings}
    \abbrev{UBM}{universal background model}
    \abbrev{UMSDA}{unsupervised multi-source domain adaptation}
    \abbrev{VAE}{variational autoencoder}
    \abbrev{VC}{Vapnik-Chervonenkis}
    \abbrev{VSIA}{Visible Spectrum Iris Artefact (dataset)}
    \abbrev{wLBP}{weighted local binary pattern}
\end{glossary}


%-------------------------------------------------------------------------
%	DEDICATORY
%-------------------------------------------------------------------------

%\begin{dedicatory}
%	This thesis is dedicated to my family, for their love and support.
%\end{dedicatory}
