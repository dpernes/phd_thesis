\chapter*{Notation and Conventions}
\label{notation}

% Spacing between notation sections
\newlength{\notationgap}
\setlength{\notationgap}{1pc}

In this section, we describe the notation adopted in this thesis. We mostly follow the notation proposed by~\citet{Goodfellow2016}, which is also the recommended one by the International Conference on Learning Representations (ICLR).\todo{This section must be reviewed after the thesis is complete. Some definitions might be missing and others are not even used, so those should be removed.}

\vspace{\notationgap}
% Need to use minipage to keep title of table on same page as table
\begin{minipage}{\textwidth}
	% This is a hack to put a little title over the table
	% We cannot use "\section*", etc., they appear in the table of contents.
	% tocdepth does not work on this chapter.
	\centerline{\bf Numbers and Arrays}
	\bgroup
	% The \arraystretch definition here increases the space between rows in the table,
	% so that \displaystyle math has more vertical space.
	\def\arraystretch{1.5}
	\begin{tabular}{cp{3.25in}}
		$\displaystyle a$ & A scalar (integer or real)\\
		$\displaystyle \va$ & A vector\\
		$\displaystyle \mA$ & A matrix\\
		$\displaystyle \tA$ & A tensor\\
		$\displaystyle \mI_n$ & Identity matrix with $n$ rows and $n$ columns\\
		$\displaystyle \mI$ & Identity matrix with dimensionality implied by context\\
		$\displaystyle \text{diag}(\va)$ & A square, diagonal matrix with diagonal entries given by $\va$\\
		$\displaystyle \ra$ & A scalar random variable\\
		$\displaystyle \rva$ & A vector-valued random variable\\
		$\displaystyle \rmA$ & A matrix-valued random variable\\
	\end{tabular}
	\egroup
	\index{Scalar}
	\index{Vector}
	\index{Matrix}
	\index{Tensor}
\end{minipage}

\vspace{\notationgap}
\begin{minipage}{\textwidth}
	\centerline{\bf Sets and Graphs}
	\bgroup
	\def\arraystretch{1.5}
	\begin{tabular}{cp{3.25in}}
		$\displaystyle \R$ & The set of real numbers \\
		% NOTE: do not use \R^+, because it is ambiguous whether:
		% - It includes 0
		% - It includes only real numbers, or also infinity.
		% We usually do not include infinity, so we may explicitly write
		% [0, \infty) to include 0
		% (0, \infty) to not include 0
		$\displaystyle \{0, 1\}$ & The set containing 0 and 1 \\
		$\displaystyle \{1, \dots, n \}$ & The set of all integers between $1$ and $n$\\
		$\displaystyle [a, b]$ & The real interval including $a$ and $b$\\
		$\displaystyle (a, b]$ & The real interval excluding $a$ but including $b$\\
		$\displaystyle \gA \backslash \gB$ & Set subtraction, i.e.\ the set containing the elements of $\gA$ that are not in $\gB$\\
		$\displaystyle \gG$ & A graph\\
		$\displaystyle \parents_\gG(\rx)$ & The parents of node $\rx$ in $\gG$
	\end{tabular}
	\egroup
	\index{Scalar}
	\index{Vector}
	\index{Matrix}
	\index{Tensor}
	\index{Graph}
	\index{Set}
\end{minipage}

\vspace{\notationgap}
\begin{minipage}{\textwidth}
	\centerline{\bf Indexing}
	\bgroup
	\def\arraystretch{1.5}
	\begin{tabular}{cp{3.25in}}
		$\displaystyle \eva_i$ & Element $i$ of vector $\va$, with indexing starting at 1 \\
		$\displaystyle \emA_{i,j}$ & Element $i, j$ of matrix $\mA$ \\
		$\displaystyle \erva_i$ & Element $i$ of the random vector $\rva$ \\
	\end{tabular}
	\egroup
\end{minipage}

\vspace{\notationgap}
\begin{minipage}{\textwidth}
	\centerline{\bf Linear Algebra Operations}
	\bgroup
	\def\arraystretch{1.5}
	\begin{tabular}{cp{3.25in}}
		$\displaystyle \mA \transp$ & Transpose of matrix $\mA$ \\
		$\displaystyle \mA \odot \mB $ & Element-wise (Hadamard) product of $\mA$ and $\mB$ \\
		% Wikipedia uses \circ for element-wise multiplication but this could be confused with function composition
		$\displaystyle \mathrm{det}(\mA)$ & Determinant of $\mA$ \\
	\end{tabular}
	\egroup
	\index{Transpose}
	\index{Element-wise product|see {Hadamard product}}
	\index{Hadamard product}
	\index{Determinant}
\end{minipage}

\vspace{\notationgap}
\begin{minipage}{\textwidth}
	\centerline{\bf Calculus}
	\bgroup
	\def\arraystretch{1.5}
	\begin{tabular}{cp{3.25in}}
		% NOTE: the [2ex] on the next line adds extra height to that row of the table.
		% Without that command, the fraction on the first line is too tall and collides
		% with the fraction on the second line.
		$\displaystyle\frac{d y} {d x}$ & Derivative of $y$ with respect to $x$\\ [2ex]
		$\displaystyle \frac{\partial y} {\partial x} $ & Partial derivative of $y$ with respect to $x$ \\
		$\displaystyle \nabla_\vx y $ & Gradient of $y$ with respect to $\vx$ \\
		$\displaystyle \nabla_\mX y $ & Matrix derivatives of $y$ with respect to $\mX$ \\
		$\displaystyle \int f(\vx) \d\vx $ & Definite integral over the entire domain of $\vx$ \\
		$\displaystyle \int_\gX f(\vx) \d\vx$ & Definite integral with respect to $\vx$ over the set $\gX$ \\
	\end{tabular}
	\egroup
	\index{Derivative}
	\index{Integral}
	\index{Jacobian matrix}
	\index{Hessian matrix}
\end{minipage}

\vspace{\notationgap}
\begin{minipage}{\textwidth}
	\centerline{\bf Probability and Information Theory}
	\bgroup
	\def\arraystretch{1.5}
	\begin{tabular}{cp{3.25in}}
		$\displaystyle \ra \bot \rb$ & The random variables $\ra$ and $\rb$ are independent\\
		$\displaystyle \ra \bot \rb \mid \rc $ & They are conditionally independent given $\rc$\\
		$\displaystyle p(\rx)$ & A probability distribution over the random variable $\rx$\\
		$\displaystyle \rx \sim p$ & Random variable $\rx$ has distribution $p$\\% so thing on left of \sim should always be a random variable, with name beginning with \r
        $\displaystyle \mathrm{Supp}(\rx)$ & Support of the random variable $\rx$\\
		$\displaystyle  \E_{\rx\sim p} [ f(\rx) ]$ & Expectation of $f(\rx)$ with respect to $p(\rx)$ \\
		$\displaystyle H(\rx) $ & Shannon entropy of the random variable $\rx$\\
		$\displaystyle \KL ( p \Vert q ) $ & Kullback-Leibler divergence of p and q \\
		$\displaystyle \mathcal{N} ( \rvx ; \vmu , \mSigma)$ & Gaussian distribution %
		over $\rvx$ with mean $\vmu$ and covariance $\mSigma$ \\
	\end{tabular}
	\egroup
	\index{Independence}
	\index{Conditional independence}
	\index{Variance}
	\index{Covariance}
	\index{Kullback-Leibler divergence}
	\index{Shannon entropy}
\end{minipage}

\vspace{\notationgap}
\begin{minipage}{\textwidth}
	\centerline{\bf Functions}
	\bgroup
	\def\arraystretch{1.5}
	\begin{tabular}{cp{3.25in}}
		$\displaystyle f: \gA \rightarrow \gB$ & The function $f$ with domain $\gA$ and range $\gB$\\
		$\displaystyle f \circ g $ & Composition of the functions $f$ and $g$ \\
		$\displaystyle f(\vx ; \vtheta) $ & A function of $\vx$ parametrized by $\vtheta$.
		(Sometimes we write $f(\vx)$ and omit the argument $\vtheta$ to lighten notation) \\
		$\displaystyle \log x$ & Natural logarithm of $x$ \\
		$\displaystyle \sigma(x)$ & Logistic sigmoid, $\displaystyle \frac{1} {1 + \exp(-x)}$ \\
		$\displaystyle || \vx ||_p $ & $\normlp$ norm of vector $\vx$ \\
		$\displaystyle || \vx || $ & $\normltwo$ norm of vector $\vx$ \\
        $\displaystyle || \mX || $ & Frobenius norm of matrix $\mX$ \\
		$\displaystyle x^+$ & Positive part of $x$, i.e.\ $\max(0,x)$\\
		$\displaystyle \1_\mathrm{condition}$ & is 1 if the condition is true, 0 otherwise\\
	\end{tabular}
	\egroup
	\index{Sigmoid}
	\index{Softplus}
	\index{Norm}
\end{minipage}

Sometimes we use a function $f$ whose argument is a scalar but apply
it to a vector, matrix, or tensor: $f(\vx)$, $f(\mX)$, or $f(\tX)$.
This denotes the application of $f$ to the
array element-wise. For example, if $\tC = \sigma(\tX)$, then $\etC_{i,j,k} = \sigma(\etX_{i,j,k})$
for all valid values of $i$, $j$ and $k$.

